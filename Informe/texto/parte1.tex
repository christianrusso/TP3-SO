\section{Implementacion Map-Reduce}

\subsection{Encontrar el subreddit con mayor score promedio}

Para realizar este análisis se implemento una función \textbf{Map}, la cual utiliza como clave el campo \textit{subreddit} del post y a dicha clave asocia la tupla formada por el campo \textit{score} del post y un uno para contabilizar el post.
Luego, la función \textbf{Reduce} contabiliza la cantidad de post de un mismo subreddit y calcula la sumatoria de los scores de estos, retornando ambos en una tupla.
Finalmente, la función \textbf{Finalize} se encarga de calcular el promedio con los dos valores anteriores.

Luego, ejecutamos en Mongo los siguientes comandos para obtener el resultado buscado:
\begin{lstlisting}
use reddit
db.default.find().sort({"value":-1).limit(1)
\end{lstlisting}

Simplemente ordenamos los resultados en orden descendente en función del campo \textit{value}, el cual contiene el score promedio de cada subreddit, y nos quedamos con el primero, es decir, el máximo.

El resultado obtenido es que el subreddit \textit{GirlGamers} es el que tiene mayor score promedio, con un valor de 2483.


\subsection{Encontrar los doce títulos con mayor score de la colección de posts con al menos 2000 votos}

La función \textbf{Map} implementada utiliza como clave el campo title del post y como valor asociado utiliza una tupla formada por los campos total\_votes y score de dicho post.
Luego, la función \textbf{Reduce} calcula la sumatoria de los votos totales y la sumatoria de los scores y retorna ambos resultados en una tupla.
Finalmente, la función \textbf{Finalize} verifica si titulo tiene mas de 2000 votos y, en caso afirmativo, retorna la sumatoria de los scores. En caso contrario retorna cero.

Luego, utilizamos los siguientes comandos de Mongo:
\begin{lstlisting}
use reddit
db.default.find().sort({"value":-1).limit(12)
\end{lstlisting}

Utilizamos casi los mismos comandos que en el punto anterior, solo que aqui ordenamos de forma descendente el campo \textit{value}, el cual contiene el score de cada titulo, y nos quedamos con los primeros doce.

Los doce títulos con mayor score y, al menos, 2000 votos son:
\begin{lstlisting}
The Bus Knight
Haters gonna hate
So my little cousin posted on FB that he was bored and gave everyone his new phone number... (pic)
My friend calls him &quot;Mr Ridiculously Photogenic Guy&quot;
This is called humanity.
Genius
Seems legit.
President Obama's new campaign poster
Poster ad for the Canadian Paralympics
Fuck the police
Soon...
I'm sorry pinata bro
\end{lstlisting}


\subsection{Para los diez mejores scores, calcular la cantidad de comentarios en promedio por sumisión}

La función \textbf{Map} implementada para este ejercicio utiliza como clave el campo \textit{score} del post y le asocia la tupla formada por un uno (para contabilizar la sumisión) y el campo \textit{number\_of\_comments} de dicho post.
Luego, la función \textbf{Reduce} retorna una tupla formada por la sumatoria de unos (cantidad de sumisiones) y la sumatoria del numero de comentarios para cada score (cantidad de comentarios).
Finalmente, la función \textbf{Finalize} se encarga de calcular la cantidad de comentarios en promedio realizando la división de los dos valores retornados por Reduce.

Luego, utilizamos en Mongo los siguientes comandos:
\begin{lstlisting}
use reddit
db.default.find().sort({"_id": -1).limit(10)
\end{lstlisting}

Lo que hacemos es ordenar los resultados en forma descendente por su score (el cual recordemos que utilizamos como clave) y luego no quedamos con los 10 mejores.

El resultado obtenido es:
\begin{lstlisting}
Score   Cantidad de comentarios en promedio por sumision
20570                      1463 
12333                      1612
11908                      2681
10262                      1514
 8935                       480
 8835                      1716
 8699                       934
 8241                       571
 7297                      1110
 6741                      2204
\end{lstlisting}


\subsection{Entre los usuarios con a lo sumo 5 sumisiones, encontrar el que posee mayor cantidad de upvotes}

La función \textbf{Map} utiliza el campo \textit{username} como clave y le asocia la tupla formada por un uno (para contabilizar la sumisión) y el campo \textit{upvotes} del post.
Luego, la función Reduce se encarga de realizar la sumatoria de sumisiones y upvotes para cada usuario y, finalmente, la función Finalize se encarga de retornar el numero de upvotes para aquellos usuarios que tengan a lo sumo 5 sumisiones. Para aquellos usuarios que tienen mas de 5 sumisiones retorna cero.

En Mongo utilizamos los comandos:
\begin{lstlisting}
use reddit
db.default.find().sort({"value":-1).limit(1)
\end{lstlisting}

De la misma forma que en el primer ejercicio, obtenemos al usuario con mayor cantidad de upvotes y a lo sumo 5 sumisiones. El usuario encontrado es \textit{lepry} con 90396 upvotes.


\subsection{Para todos los subreddit que poseen un score entre 280 y 300, indicar la cantidad de palabras presentes en sus títulos}

La función \textbf{Map} utiliza como clave el campo \textit{subreddit} del post y le asocia la tupla formada por el campo \textit{score} del post y un entero indicando la cantidad de palabras en el titulo del post \textit{(calculado como this.title.split(" ").length)}.
Luego, la función \textbf{Reduce} lleva a cabo la sumatoria de la cantidad de palabras en el titulo y la sumatoria de los score para cada subreddit.
Finalmente, la función \textbf{Finalize} determina si el subreddit tiene un score entre 280 y 300. En caso afirmativo retorna la sumatoria de la cantidad de palabras en sus títulos. En caso contrario retorna -1.

Luego, utilizamos los siguientes comandos en Mongo:
\begin{lstlisting}
use reddit
db.default.find({"value": {$ne: -1}})
\end{lstlisting}

Con este comando filtramos los resultados para solo quedarnos con aquellos que nos interesan, es decir, para los cuales \textbf{Finalize} no retorno -1 y que, por lo tanto, tienen un score entre 280 y 300.

El resultado obtenido es que los siguiente subreddit son los que tienen un score entre 280 y 300 (entre paréntesis colocamos la cantidad de palabras presentes en sus títulos):
\begin{lstlisting}
Subreddit             Cantidad de palabras presentes en sus titulos
Feminism                                       6
Firearms                                      24
HeroesofNewerth                                6
Sexy                                          13
TheRealZachAnner                               4
ragecomics                                     4
xkcd                                           4
\end{lstlisting}